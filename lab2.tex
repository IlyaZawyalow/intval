\documentclass[a4paper,14pt]{extarticle}
\usepackage[utf8]{inputenc}
\usepackage[russian]{babel}
\usepackage{amsmath,amsfonts,amssymb}
\usepackage{geometry}
\usepackage{graphicx}
\usepackage{url}
\usepackage[nottoc, notlof, notlot]{tocbibind}
\usepackage{listings}

\geometry{top=2cm, bottom=2cm, left=3cm, right=1.5cm}
\lstset{
    inputencoding=utf8,
    extendedchars=\true,
    literate={-}{{-}}1 {\->}{{\texttt{->}}}2 {_}{{\_}}1,
    basicstyle=\ttfamily,
    frame=single,
    captionpos=b,
}

\begin{document}

\begin{titlepage}
    \begin{center}
        Санкт-Петербургский политехнический университет Петра Великого\\
        Физико-механический институт\\[4cm]
        
        \textbf{Отчёт по лабораторным работам 1 и 2}\\[0.5cm]
        по дисциплине «Интервальный анализ»\\[0.5cm]
        \textbf{«Калибровка чипа быстродействующей аналоговой памяти PCI DRS4»}\\[4cm]

        
        \begin{flushleft}
            Выполнил\\
            студент гр. 5040102/30201 \hfill Завьялов И.В. \hfill \rule{3cm}{0.1pt} \\[1.5cm]
            Проверил\\
            доцент, к.ф.-м.н. \hfill \hspace{1.9cm} Баженов А.Н. \hfill \rule{3cm}{0.1pt} \\[1.5cm]
        \end{flushleft}
        
        \vfill
        Санкт-Петербург\\
        2024
    \end{center}
\end{titlepage}

\newpage

\tableofcontents

\newpage
\section{Постановка задачи}

В рамках данной лабораторной работы рассматривается калибровка чипа быстродействующей аналоговой памяти PCI DRS4, который используется в области солнечной энергетики. Этот чип оснащён 8 каналами, каждый из которых содержит 1024 ячейки памяти. Каждая ячейка включает конденсаторы для хранения заряда и электронные ключи для записи сигналов и считывания напряжений через аналогово-цифровой преобразователь (АЦП). Ячейки объединены в кольцевые буферы, которые обеспечивают эффективное управление записями и чтением данных.

\subsection*{Процесс калибровки}
\begin{enumerate}
    \item \textbf{Ввод напряжения:} В чип подаётся известное напряжение $X$.
    \item \textbf{Считывание значений:} Для каждого значения $X$ операция повторяется 100 раз, и считываются полученные значения $Y$ через АЦП.
    \item \textbf{Линейная регрессия:} Предполагается линейная зависимость между $X$ и $Y$ в виде:
    \[
    Y = B_0 \cdot X + B_1
    \]
    Проводится линейная регрессия для определения коэффициентов $B_0$ и $B_1$.
\end{enumerate}

\section{Теория}
\subsection{Первый подход: нахождение $\arg\max(\text{Tol})$}


Измеренные значения $Y$ имеют погрешности, поэтому каждое значение $Y$ следует рассматривать не как точку, а как интервал неопределённости. Центральное значение интервала совпадает с измеренным $Y$, а радиус интервала равен:
\[
\epsilon = \frac{1}{16535}.
\]
Это отражает точность АЦП, используемого в чипе.


Поскольку показания с разных ячеек независимы, можно рассмотреть любую ячейку из всех $8192$ (8 каналов по 1024 ячейки). Для выбранной ячейки у нас есть 100 пар значений $(X, Y)$, где $X$ находится в диапазоне $[-0.5, 0.5]$, а $Y$ представлено интервалами с шириной $\frac{2}{16535}$.

Для нахождения точечной оценки коэффициентов калибровки используется функционал Tol, который измеряет степень соответствия модели данным с учётом погрешностей:
\[
\text{Tol}(x, A, B) = \min \left( \text{rad}(b_i) - \left| \text{mid}(b_i) - (A \cdot x + B) \right| \right),
\]
где:
\begin{itemize}
    \item $\text{rad}(b_i)$ — радиус интервала измерения $Y_i$,
    \item $\text{mid}(b_i)$ — центральное значение интервала измерения $Y_i$,
    \item $A$ и $B$ — текущие оценки коэффициентов $B_0$ и $B_1$.
\end{itemize}

Допусковое множество решений определяется как набор всех параметров $x$, для которых функционал $\text{Tol}$ положителен:
\[
\{x \in \mathbb{R}^n \mid \text{Tol}(x, A, B) \geq 0\}.
\]
Это означает, что модель полностью согласуется с измеренными данными, учитывая погрешности.

В случаях, когда система уравнений оказывается несовместной, необходимо увеличить радиусы интервалов измерений до тех пор, пока система не станет совместной. Это достигается путём "расширения" интервалов $Y$, что позволяет функционалу $\text{Tol}$ стать неотрицательным для всех измерений.

\subsection{Второй метод: нахождение оценки при помощи твинной арифметики}

\subsubsection*{Недостатки первого метода}
\begin{itemize}
    \item \textbf{Расширение интервалов:} При увеличении интервалов возникает значительная погрешность, так как интервалы растягиваются в обе стороны.
    \item \textbf{Точечная оценка:} Первый метод предоставляет лишь точечную оценку коэффициентов $B_0$ и $B_1$, что может быть недостаточно информативно.
\end{itemize}

Используется твинная арифметика для более точной оценки параметров. Основные шаги метода:

\paragraph{1. Группировка данных:}
\begin{itemize}
    \item Для каждого значения $X$ собирается 100 значений $Y$.
    \item Для каждой группы значений строится боксплот Тьюки, который позволяет определить внешние и внутренние оценки значений $Y$.
\end{itemize}

\paragraph{2. Построение интервалов:}
На основе боксплотов формируются интервалы для $Y$ для каждого $X$.

\paragraph{3. Распознающий функционал Tol:}
Функционал Tol создаётся на основе новых интервалов:
\[
\text{Tol}(x, A, B) = \min \left( \text{rad}(b_i) - \left| \text{mid}(b_i) - (A \cdot x + B) \right| \right),
\]
где $\text{rad}(b_i)$ и $\text{mid}(b_i)$ определяют радиус и центральное значение интервала $Y_i$, а $A$ и $B$ — текущие оценки коэффициентов.

\paragraph{4. Условия:}
\begin{itemize}
    \item Если $\text{Tol}(\arg\max(\text{Tol})) = 0$, то найденные коэффициенты считаются оптимальными.
    \item Если $\text{Tol} > 0$, то возвращается множество коэффициентов, удовлетворяющих этому условию.
    \item Если $\text{Tol} < 0$, то соответствующие строки из матрицы $A$ и вектора $b$ исключаются, и процесс повторяется для достижения совместимости системы.
\end{itemize}

\section{Результаты}
В ходе эксперимента каждому датчику в чипе были присвоены соответствующие коэффициенты $B_0$ и $B_1$ в зависимости от его положения (координаты канала и ячейки).

Результаты для датчиков с координатами (3, 600) и (3, 500) изображены на рисунках 1-6.
\clearpage
\begin{figure}[htbp]
    \centering
    \includegraphics[width=0.6\textwidth]{f1.png}
    \caption{Регрессионная прямая для датчика (3, 600) полученная методом $\arg\max(\text{Tol})$}
    \label{fig:hamiltonianGraph}
\end{figure}

\begin{figure}[htbp]
    \centering
    \includegraphics[width=0.6\textwidth]{f3.png}
    \caption{Регрессионная прямая для датчика (3, 600) полученная вторым методом}
    \label{fig:hamiltonianGraph}
\end{figure}
\clearpage
\begin{figure}[htbp]
    \centering
    \includegraphics[width=0.6\textwidth]{f4.png}
    \caption{Tol, Uni и argmaxTol для датчика (3, 600)}
    \label{fig:hamiltonianGraph}
\end{figure}

\begin{figure}[htbp]
    \centering
    \includegraphics[width=0.6\textwidth]{f5.png}
    \caption{Регрессионная прямая для датчика (3, 500) полученная методом $\arg\max(\text{Tol})$}
    \label{fig:hamiltonianGraph}
\end{figure}
\clearpage
\begin{figure}[htbp]
    \centering
    \includegraphics[width=0.6\textwidth]{f6.png}
    \caption{Регрессионная прямая для датчика (3, 500) полученная вторым методом}
    \label{fig:hamiltonianGraph}
\end{figure}

\begin{figure}[htbp]
    \centering
    \includegraphics[width=0.6\textwidth]{f7.png}
    \caption{Tol, Uni и argmaxTol для датчика (3, 500)}
    \label{fig:hamiltonianGraph}
\end{figure}
\clearpage
Анализ представленных графиков и результатов показал, что оба метода калибровки выполняют свою работу корректно. Хотя результаты двух методов близки, небольшие различия могут быть обусловлены особенностями каждого подхода:
\begin{itemize}
    \item \textbf{Первый метод:} Позволяет быстро получить точечные оценки коэффициентов, но может приводить к значительным погрешностям при расширении интервалов.
    \item \textbf{Второй метод:} Обеспечивает более точные оценки и устойчивость к выбросам благодаря использованию твинной арифметики и более аккуратному подходу к обработке данных.
\end{itemize}
Кроме того, различия в количестве выбросов между датчиками указывают на необходимость дополнительного анализа причин таких отклонений, возможно, связанных с техническими характеристиками конкретных ячеек или каналов чипа.

\section{Заключение}
В ходе лабораторной работы были изучены и реализованы два метода калибровки чипа PCI DRS4 с использованием интервального анализа. Оба подхода показали свою эффективность в условиях погрешностей измерений, но второй метод с твинной арифметикой продемонстрировал большую точность и устойчивость к выбросам. Эти методы позволяют учитывать неопределённость данных и обеспечивают надёжную калибровку параметров модели, что важно для точной работы аналогово-цифровых преобразователей в системах солнечной энергетики.


\section{Код и ресурсы}
Код проекта доступен в публичном репозитории на GitHub.

https://github.com/IlyaZawyalow/intval

\end{document}